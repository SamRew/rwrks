\documentclass{beamer}\usepackage[]{graphicx}\usepackage[]{color}
%% maxwidth is the original width if it is less than linewidth
%% otherwise use linewidth (to make sure the graphics do not exceed the margin)
\makeatletter
\def\maxwidth{ %
  \ifdim\Gin@nat@width>\linewidth
    \linewidth
  \else
    \Gin@nat@width
  \fi
}
\makeatother

\definecolor{fgcolor}{rgb}{0.345, 0.345, 0.345}
\newcommand{\hlnum}[1]{\textcolor[rgb]{0.686,0.059,0.569}{#1}}%
\newcommand{\hlstr}[1]{\textcolor[rgb]{0.192,0.494,0.8}{#1}}%
\newcommand{\hlcom}[1]{\textcolor[rgb]{0.678,0.584,0.686}{\textit{#1}}}%
\newcommand{\hlopt}[1]{\textcolor[rgb]{0,0,0}{#1}}%
\newcommand{\hlstd}[1]{\textcolor[rgb]{0.345,0.345,0.345}{#1}}%
\newcommand{\hlkwa}[1]{\textcolor[rgb]{0.161,0.373,0.58}{\textbf{#1}}}%
\newcommand{\hlkwb}[1]{\textcolor[rgb]{0.69,0.353,0.396}{#1}}%
\newcommand{\hlkwc}[1]{\textcolor[rgb]{0.333,0.667,0.333}{#1}}%
\newcommand{\hlkwd}[1]{\textcolor[rgb]{0.737,0.353,0.396}{\textbf{#1}}}%

\usepackage{framed}
\makeatletter
\newenvironment{kframe}{%
 \def\at@end@of@kframe{}%
 \ifinner\ifhmode%
  \def\at@end@of@kframe{\end{minipage}}%
  \begin{minipage}{\columnwidth}%
 \fi\fi%
 \def\FrameCommand##1{\hskip\@totalleftmargin \hskip-\fboxsep
 \colorbox{shadecolor}{##1}\hskip-\fboxsep
     % There is no \\@totalrightmargin, so:
     \hskip-\linewidth \hskip-\@totalleftmargin \hskip\columnwidth}%
 \MakeFramed {\advance\hsize-\width
   \@totalleftmargin\z@ \linewidth\hsize
   \@setminipage}}%
 {\par\unskip\endMakeFramed%
 \at@end@of@kframe}
\makeatother

\definecolor{shadecolor}{rgb}{.97, .97, .97}
\definecolor{messagecolor}{rgb}{0, 0, 0}
\definecolor{warningcolor}{rgb}{1, 0, 1}
\definecolor{errorcolor}{rgb}{1, 0, 0}
\newenvironment{knitrout}{}{} % an empty environment to be redefined in TeX

\usepackage{alltt} 
% \usepackage{graphicx}
\usepackage{graphics}
\usepackage[T1]{fontenc}
\usepackage{verbatim}
\usepackage{etoolbox}
\usepackage{hyperref}
\usepackage{color}
\makeatletter
\preto{\@verbatim}{\topsep=-6pt \partopsep=-6pt }
\makeatother
%\usepackage{fix-cm}
\setbeamercovered{transparent}


\renewcommand{\ni}{\noindent}


% \SweaveOpts{cache=TRUE, background="white"}


\title[1-Files]{01 - Files}
\subtitle{Reading Files}
% \author[E. Hare, S. VanderPlas]{Eric Hare and Susan VanderPlas}
\date{\hspace{1in}}
\institute[ISU]{Iowa State University}
\graphicspath{{figures/}}
\IfFileExists{upquote.sty}{\usepackage{upquote}}{}
\begin{document}

\begin{frame}
\maketitle
\end{frame}



\begin{frame}
\frametitle{Outline}
\begin{itemize}
\item Reading files: Excel and R\medskip
\item Packages \texttt{gdata} and \texttt{foreign} \medskip
\item Reading SAS xport files\medskip
\end{itemize}
\end{frame}


\begin{frame}
\frametitle{Data in Excel}
\begin{itemize}
\item Formats xls and csv - what's the difference?
\item File extensions xls and xlsx are proprietary Excel formats, binary files
\item csv is an extension for Comma Separated Value files. They are text files - directly readable.
\item Example: Gas prices in midwest since 1994
\end{itemize}
\end{frame}

\begin{frame}
\frametitle{Reading Files in R}
\begin{itemize}
\item Textfiles: Usually comma-separated (or tabular separated)
\end{itemize}
\begin{knitrout}\scriptsize
\definecolor{shadecolor}{rgb}{1, 1, 1}\color{fgcolor}\begin{kframe}
\begin{alltt}
\hlopt{?}\hlstd{read.csv}
\hlopt{?}\hlstd{read.table}
\end{alltt}
\end{kframe}
\end{knitrout}
\begin{knitrout}\scriptsize
\definecolor{shadecolor}{rgb}{1, 1, 1}\color{fgcolor}\begin{kframe}
\begin{alltt}
\hlstd{midwest} \hlkwb{<-} \hlkwd{read.csv}\hlstd{(}\hlstr{"http://heike.github.io/rwrks/03a-r-format/data/01-data/midwest.csv"}\hlstd{)}
\end{alltt}
\end{kframe}
\end{knitrout}
\end{frame}

\begin{frame}[fragile]
\frametitle{Gas Prices in the Midwest}
\begin{knitrout}\scriptsize
\definecolor{shadecolor}{rgb}{1, 1, 1}\color{fgcolor}\begin{kframe}
\begin{alltt}
\hlkwd{str}\hlstd{(midwest)}
\end{alltt}
\begin{verbatim}
## 'data.frame':	212 obs. of  11 variables:
##  $ Year.Month: Factor w/ 212 levels "","1994-Dec",..: 1 3 2 8 7 11 4 12 10 9 ...
##  $ Week.1    : Factor w/ 86 levels "","1-Apr","1-Aug",..: 86 1 52 18 65 69 26 10 56 31 ...
##  $ X         : Factor w/ 197 levels "","0.905","0.918",..: 197 1 19 7 12 13 21 29 42 31 ...
##  $ Week.2    : Factor w/ 86 levels "","10-Apr","10-Aug",..: 86 1 28 78 41 45 2 70 32 7 ...
##  $ X.1       : Factor w/ 206 levels "","0.919","0.921",..: 206 1 17 14 12 13 27 39 45 34 ...
##  $ Week.3    : Factor w/ 86 levels "","15-Apr","15-Aug",..: 86 1 52 18 65 69 26 10 56 31 ...
##  $ X.2       : Factor w/ 199 levels "","0.91","0.929",..: 199 1 11 9 9 15 28 40 38 29 ...
##  $ Week.4    : Factor w/ 85 levels "22-Apr","22-Aug",..: 85 82 51 17 64 68 25 9 55 30 ...
##  $ X.3       : Factor w/ 201 levels "0.883","0.921",..: 201 29 9 14 13 15 32 44 34 27 ...
##  $ Week.5    : Factor w/ 31 levels "","29-Apr","29-Aug",..: 31 1 1 16 1 1 1 9 1 27 ...
##  $ X.4       : Factor w/ 74 levels "","0.955","1.023",..: 74 1 1 5 1 1 1 18 1 11 ...
\end{verbatim}
\end{kframe}
\end{knitrout}

There is clearly some work to be done with the data...
\end{frame}

\begin{frame}
\frametitle{Your Turn}
\begin{itemize}
\item Have a look at the parameters of read.table (?read.table) to solve the following problems:
\item Read the first two lines of the file into an object called `midwest\_names`
\item Read everything EXCEPT the first two lines into an object called `midwest\_data`
\end{itemize}
\end{frame}

\begin{frame}[fragile]
\frametitle{Reading Excel Data}
We use gdata to accomplish this - If you are on Windows, you might need to install Strawberry Perl from http://strawberryperl.com/
\begin{knitrout}\scriptsize
\definecolor{shadecolor}{rgb}{1, 1, 1}\color{fgcolor}\begin{kframe}
\begin{alltt}
\hlkwd{library}\hlstd{(gdata)}
\hlstd{midwest2} \hlkwb{<-} \hlkwd{read.xls}\hlstd{(}\hlstr{"http://heike.github.io/rwrks/03a-r-format/data/01-data/midwest.xls"}\hlstd{)}

\hlkwd{head}\hlstd{(midwest2)}
\end{alltt}
\begin{verbatim}
##   Year.Month   Week.1     X   Week.2   X.1   Week.3
## 1            End Date Value End Date Value End Date
## 2   1994-Nov                                       
## 3   1994-Dec    5-Dec 1.086   12-Dec 1.057   19-Dec
## 4   1995-Jan    2-Jan 1.025    9-Jan 1.046   16-Jan
## 5   1995-Feb    6-Feb 1.045   13-Feb  1.04   20-Feb
## 6   1995-Mar    6-Mar 1.053   13-Mar 1.042   20-Mar
##     X.2   Week.4   X.3   Week.5   X.4
## 1 Value End Date Value End Date Value
## 2         28-Nov 1.122               
## 3 1.039   26-Dec 1.027               
## 4 1.031   23-Jan 1.054   30-Jan 1.055
## 5 1.031   27-Feb 1.052               
## 6 1.048   27-Mar 1.065
\end{verbatim}
\end{kframe}
\end{knitrout}
\end{frame}

\begin{frame}
\frametitle{Your Turn}
\begin{itemize}
\item Read the file `usa.xls` from the website using read.xls()
\item Investigate the structure of this object - Is the data in a clean format for working, or does some work need to be done in order to begin analyzing it?
\end{itemize}
\end{frame}

\begin{frame}
\frametitle{Package foreign}
\begin{itemize}
\item Other file formats can be read using the functions from package \texttt{foreign}
\item SPSS: read.spss
\item SAS: read.xport
\item Minitab: read.mtp
\item Systat: read.systat
\end{itemize}
\end{frame}

\begin{frame}
\frametitle{Your Turn}
\begin{itemize}
\item The NHANES (National Health and Nutrition Survey) publishes data in the SAS xport format:
http://wwwn.cdc.gov/nchs/nhanes/search/nhanes11\_12.aspx
\item Scroll to the bottom, choose one of the datasets (Demographics, Dietary, etc.). Download the Data file (XPT)
\item Use read.xport() to load the file into R
\end{itemize}
\end{frame}

\end{document}
